\section{Multiplexing}
\label{sec:multiplexing}
Okay, we've covered how data gets transmitted over a medium and how we can modulate signals to represent our data. That's cool! Unfortunately, with the busy lives we lead, we need lots of data - fast! So, how do we cram all this data into our limited bandwidth? Enter multiplexing!

Multiplexing is the technique of combining multiple signals into one signal over a shared medium. This allows us to make the most out of our available bandwidth by sending multiple data streams simultaneously.

\subsection{Time Division Multiplexing (TDM)}
\label{subsec:tdm}
Time Division Multiplexing (TDM) is a method where multiple signals share the same transmission medium by dividing the time into slots. Each signal gets its own time slot to transmit its data, and the slots are allocated in a round-robin fashion. This way, each signal gets a chance to transmit without interfering with others.

TODO

\subsection{Frequency Division Multiplexing (FDM)}
\label{subsec:fdm}
Frequency Division Multiplexing (FDM) is another technique where multiple signals are transmitted simultaneously over different frequency bands. Each signal occupies a unique frequency range, allowing them to coexist without interference. This
is commonly used in radio and television broadcasting, where different channels are assigned specific frequency bands.


TODO

\subsection{Code Division Multiple Access (CDMA)}
\label{subsec:cdma}
Code Division Multiple Access (CDMA) is a more advanced multiplexing technique that allows multiple signals to share the same frequency band simultaneously. Each signal is assigned a unique code, which is used to modulate the signal. At the receiver, the unique code is used to demodulate the signal and extract the original data. This technique is widely used in mobile communication systems, allowing multiple users to share the same frequency band without interference.

TODO: Chip codes and such